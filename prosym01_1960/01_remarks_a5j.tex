\documentclass[a5j]{jsarticle}
\pagestyle{empty}
\usepackage{ascmac}
\usepackage{enumitem}
\usepackage{url}
\usepackage[dvipdfmx]{hyperref}
\usepackage[top=20truemm,left=12truemm,right=12truemm,bottom=0truemm]{geometry}
\setlist[description]{topsep=0pt,leftmargin=0pt}
\begin{document}

\small

本 PDF ファイルは 1960 年発行の「第1回プログラミング–シンポジウム報告集」をスキャンし、項目ごとに整理して、情報処理学会電子図書館「情報学広場」に掲載するものです。

\medskip

この出版物は情報処理学会への著作権譲渡がなされていませんが、情報処理学会公式 Web サイトの \url{https://www.ipsj.or.jp/topics/Past_reports.html} に下記「過去のプログラミング・シンポジウム報告集の利用許諾について」を掲載して、権利者の捜索をおこないました。そのうえで同意をいただいたもの、お申し出のなかったものを掲載しています。

\bigskip

\begin{itembox}[l]{過去のプログラミング・シンポジウム報告集の利用許諾について}

情報処理学会発行の出版物著作権は平成12年から情報処理学会著作権規程に従い、学会に帰属することになっています。

\medskip

プログラミング・シンポジウムの報告集は、情報処理学会と設立の事情が異なるため、この改訂がシンポジウム内部で徹底しておらず、情報処理学会の他の出版物が情報学広場(=情報処理学会電子図書館)で公開されているにも拘らず、古い報告集には公開されていないものが少からずありました。

\medskip

プログラミング・シンポジウムは昭和59年に情報処理学会の一部門になりましたが、それ以前の報告集も含め、この度学会の他の出版物と同様の扱いにしたいと考えます。過去のすべての報告集の論文について、著作権者(論文を執筆された故人の相続人)を探し出して利用許諾に関する同意を頂くことは困難ですので、一定期間の権利者捜索の努力をしたうえで、著作権者が見つからない場合も論文を情報学広場に掲載させていただきたいと思います。その後、著作権者が発見され、情報学広場への掲載の継続に同意が得られなかった場合には、当該論文については、掲載を停止致します。

\medskip

この措置にご意見のある方は、プログラミング・シンポジウムの辻尚史運営委員長(\href{mailto:tsuji@math.s.chiba-u.ac.jp}{\nolinkurl{tsuji@math.s.chiba-u.ac.jp}})までお申し出ください。

\medskip

加えて、著作権者について情報をお持ちの方は事務局まで情報をお寄せくださいますようお願い申し上げます。

\medskip

\begin{description}
\item 期間:2020 年 12月 18 日 ~ 2021 年 3 月 19 日
\item 掲載日:2020 年 12 月 18 日
\end{description}

\medskip

\rightline{プログラミング・シンポジウム委員会}

\medskip

\begin{description}
\item 情報処理学会著作権規程
\item \url{https://www.ipsj.or.jp/copyright/ronbun/copyright.html}
\end{description}

\medskip

\end{itembox}

\end{document}
